%	-------------------------------------------------------------------------------
% 
%		2020년 11월 29일 일요일 첫작업
%
%
%
%
%
%
%
%	-------------------------------------------------------------------------------

%	\documentclass[12pt, a3paper, oneside]{book}
	\documentclass[12pt, a4paper, oneside]{book}
%	\documentclass[12pt, a4paper, landscape, oneside]{book}

		% --------------------------------- 페이지 스타일 지정
		\usepackage{geometry}
%		\geometry{landscape=true	}
		\geometry{top 			=10em}
		\geometry{bottom			=10em}
		\geometry{left			=8em}
		\geometry{right			=8em}
		\geometry{headheight		=4em} % 머리말 설치 높이
		\geometry{headsep		=2em} % 머리말의 본문과의 띠우기 크기
		\geometry{footskip		=4em} % 꼬리말의 본문과의 띠우기 크기
% 		\geometry{showframe}
	
%		paperwidth 	= left + width + right (1)
%		paperheight 	= top + height + bottom (2)
%		width 		= textwidth (+ marginparsep + marginparwidth) (3)
%		height 		= textheight (+ headheight + headsep + footskip) (4)	



		%	===================================================================
		%	package
		%	===================================================================
%			\usepackage[hangul]{kotex}				% 한글 사용
			\usepackage{kotex}					% 한글 사용
			\usepackage[unicode]{hyperref}			% 한글 하이퍼링크 사용

		% ------------------------------ 수학 수식
			\usepackage{amssymb,amsfonts,amsmath}	% 수학 수식 사용
			\usepackage{mathtools}				% amsmath 확장판

			\usepackage{scrextend}				% 
		

		% ------------------------------ LIST
			\usepackage{enumerate}			%
			\usepackage{enumitem}			%
			\usepackage{tablists}				%	수학문제의 보기 등을 표현하는데 사용
										%	tabenum


		% ------------------------------ table 
			\usepackage{longtable}			%
			\usepackage{tabularx}			%
			\usepackage{tabu}				%




		% ------------------------------ 
			\usepackage{setspace}			%
			\usepackage{booktabs}		% table
			\usepackage{color}			%
			\usepackage{multirow}			%
			\usepackage{boxedminipage}	% 미니 페이지
			\usepackage[pdftex]{graphicx}	% 그림 사용
			\usepackage[final]{pdfpages}		% pdf 사용
			\usepackage{framed}			% pdf 사용

			
			\usepackage{fix-cm}	
			\usepackage[english]{babel}
	
		%	=======================================================================================
		% 	tikz package
		% 	
		% 	--------------------------------- 	
			\usepackage{tikz}%
			\usetikzlibrary{arrows,positioning,shapes}
			\usetikzlibrary{mindmap}			
			

		% --------------------------------- 	page
			\usepackage{afterpage}		% 다음페이지가 나온면 어떻게 하라는 명령 정의 패키지
%			\usepackage{fullpage}			% 잘못 사용하면 다 흐트러짐 주의해서 사용
%			\usepackage{pdflscape}		% 
			\usepackage{lscape}			%	 


			\usepackage{blindtext}
	
		% --------------------------------- font 사용
			\usepackage{pifont}				%
			\usepackage{textcomp}
			\usepackage{gensymb}
			\usepackage{marvosym}



		% Package --------------------------------- 

			\usepackage{tablists}				%


		% Package --------------------------------- 
			\usepackage[framemethod=TikZ]{mdframed}				% md framed package
			\usepackage{smartdiagram}								% smart diagram package



		% Package ---------------------------------    연습문제 

			\usepackage{exsheets}				%

			\SetupExSheets{solution/print=true}
			\SetupExSheets{question/type=exam}
			\SetupExSheets[points]{name=point,name-plural=points}


		% --------------------------------- 페이지 스타일 지정

		\usepackage[Sonny]		{fncychap}

			\makeatletter
			\ChNameVar	{\Large\bf}
			\ChNumVar	{\Huge\bf}
			\ChTitleVar		{\Large\bf}
			\ChRuleWidth	{0.5pt}
			\makeatother

%		\usepackage[Lenny]		{fncychap}
%		\usepackage[Glenn]		{fncychap}
%		\usepackage[Conny]		{fncychap}
%		\usepackage[Rejne]		{fncychap}
%		\usepackage[Bjarne]	{fncychap}
%		\usepackage[Bjornstrup]{fncychap}

		\usepackage{fancyhdr}
		\pagestyle{fancy}
		\fancyhead{} % clear all fields
		\fancyhead[LO]{\footnotesize \leftmark}
		\fancyhead[RE]{\footnotesize \leftmark}
		\fancyfoot{} % clear all fields
		\fancyfoot[LE,RO]{\large \thepage}
		%\fancyfoot[CO,CE]{\empty}
		\renewcommand{\headrulewidth}{1.0pt}
		\renewcommand{\footrulewidth}{0.4pt}
	
	
	
		%	--------------------------------------------------------------------------------------- 
		% 	tritlesec package
		% 	
		% 	
		% 	------------------------------------------------------------------ section 스타일 지정
	
			\usepackage{titlesec}
		
		% 	----------------------------------------------------------------- section 글자 모양 설정
			\titleformat*{\section}					{\large\bfseries}
			\titleformat*{\subsection}				{\normalsize\bfseries}
			\titleformat*{\subsubsection}			{\normalsize\bfseries}
			\titleformat*{\paragraph}				{\normalsize\bfseries}
			\titleformat*{\subparagraph}				{\normalsize\bfseries}
	
		% 	----------------------------------------------------------------- section 번호 설정
			\renewcommand{\thepart}				{\arabic{part}.}
			\renewcommand{\thesection}				{\arabic{section}.}
			\renewcommand{\thesubsection}			{\thesection\arabic{subsection}.}
			\renewcommand{\thesubsubsection}		{\thesubsection\arabic{subsubsection}}
			\renewcommand\theparagraph 			{$\blacksquare$ \hspace{3pt}}

		% 	----------------------------------------------------------------- section 페이지 나누기 설정
			\let\stdsection\section
			\renewcommand\section{\newpage\stdsection}



		%	--------------------------------------------------------------------------------------- 
		% 	\titlespacing*{commandi} {left} {before-sep} {after-sep} [right-sep]		
		% 	left
		%	before-sep		:  수직 전 간격
		% 	after-sep	 	:  수직으로 후 간격
		%	right-sep

			\titlespacing*{\section} 			{0pt}{1.0em}{1.0em}
			\titlespacing*{\subsection}	  		{0ex}{1.0em}{1.0em}
			\titlespacing*{\subsubsection}		{0ex}{1.0em}{1.0em}
			\titlespacing*{\paragraph}			{0em}{1.5em}{1.0em}
			\titlespacing*{\subparagraph}		{4em}{1.0em}{1.0em}
	
		%	\titlespacing*{\section} 			{0pt}{0.0\baselineskip}{0.0\baselineskip}
		%	\titlespacing*{\subsection}	  		{0ex}{0.0\baselineskip}{0.0\baselineskip}
		%	\titlespacing*{\subsubsection}		{6ex}{0.0\baselineskip}{0.0\baselineskip}
		%	\titlespacing*{\paragraph}			{6pt}{0.0\baselineskip}{0.0\baselineskip}
	

		% --------------------------------- recommend		섹션별 페이지 상단 여백
		\newcommand{\SectionMargin}				{\newpage  \null \vskip 2cm}
		\newcommand{\SubSectionMargin}			{\newpage  \null \vskip 2cm}
		\newcommand{\SubSubSectionMargin}		{\newpage  \null \vskip 2cm}


		%	--------------------------------------------------------------------------------------- 
		% 	toc 설정  - table of contents
		% 	
		% 	
		% 	----------------------------------------------------------------  문서 기본 사항 설정
			\setcounter{secnumdepth}{4} 		% 문단 번호 깊이
			\setcounter{tocdepth}{2} 			% 문단 번호 깊이 - 목차 출력시 출력 범위

			\setlength{\parindent}{0cm} 		% 문서 들여 쓰기를 하지 않는다.


		%	--------------------------------------------------------------------------------------- 
		% 	mini toc 설정
		% 	
		% 	
		% 	--------------------------------------------------------- 장의 목차  minitoc package
			\usepackage{minitoc}

			\setcounter{minitocdepth}{1}    	%  Show until subsubsections in minitoc
%			\setlength{\mtcindent}{12pt} 	% default 24pt
			\setlength{\mtcindent}{24pt} 	% default 24pt

		% 	--------------------------------------------------------- part toc
		%	\setcounter{parttocdepth}{2} 	%  default
			\setcounter{parttocdepth}{0}
		%	\setlength{\ptcindent}{0em}		%  default  목차 내용 들여 쓰기
			\setlength{\ptcindent}{0em}         


		% 	--------------------------------------------------------- section toc

			\renewcommand{\ptcfont}{\normalsize\rm} 		%  default
			\renewcommand{\ptcCfont}{\normalsize\bf} 	%  default
			\renewcommand{\ptcSfont}{\normalsize\rm} 	%  default


		%	=======================================================================================
		% 	tocloft package
		% 	
		% 	------------------------------------------ 목차의 목차 번호와 목차 사이의 간격 조정
			\usepackage{tocloft}

		% 	------------------------------------------ 목차의 내어쓰기 즉 왼쪽 마진 설정
			\setlength{\cftsecindent}{2em}			%  section

		% 	------------------------------------------ 목차의 목차 번호와 목차 사이의 간격 조정
			\setlength{\cftsecnumwidth}{2em}		%  section





		%	=======================================================================================
		% 	flowchart  package
		% 	
		% 	------------------------------------------ 목차의 목차 번호와 목차 사이의 간격 조정
			\usepackage{flowchart}
			\usetikzlibrary{arrows}


		%	=======================================================================================
		% 		makeindex package
		% 	
		% 	------------------------------------------ 목차의 목차 번호와 목차 사이의 간격 조정
%			\usepackage{makeindex}
%			\usepackage{makeidy}


		%	=======================================================================================
		% 		각주와 미주
		% 	

		\usepackage{endnotes} %미주 사용


		%	=======================================================================================
		% 	줄 간격 설정
		% 	
		% 	
		% 	--------------------------------- 	줄간격 설정
			\doublespace
%			\onehalfspace
%			\singlespace
		
		

	% 	============================================================================== itemi Global setting

	
		%	-------------------------------------------------------------------------------
		%		Vertical spacing
		%	-------------------------------------------------------------------------------
			\setlist[itemize]{topsep=0.0em}			% 상단의 여유치
			\setlist[itemize]{partopsep=0.0em}			% 
			\setlist[itemize]{parsep=0.0em}			% 
%			\setlist[itemize]{itemsep=0.0em}			% 
			\setlist[itemize]{noitemsep}				% 
			
		%	-------------------------------------------------------------------------------
		%		Horizontal spacing
		%	-------------------------------------------------------------------------------
			\setlist[itemize]{labelwidth=1em}			%  라벨의 표시 폭
			\setlist[itemize]{leftmargin=8em}			%  본문 까지의 왼쪽 여백  - 4em
			\setlist[itemize]{labelsep=3em} 			%  본문에서 라벨까지의 거리 -  3em
			\setlist[itemize]{rightmargin=0em}			% 오른쪽 여백  - 4em
			\setlist[itemize]{itemindent=0em} 			% 점 내민 거리 label sep 과 같은면 점위치 까지 내민다
			\setlist[itemize]{listparindent=3em}		% 본문 드려쓰기 간격
	
	
			\setlist[itemize]{ topsep=0.0em, 			%  상단의 여유치
						partopsep=0.0em, 		%  
						parsep=0.0em, 
						itemsep=0.0em, 
						labelwidth=1em, 
						leftmargin=2.5em,
						labelsep=2em,			%  본문에서 라벨 까지의 거리
						rightmargin=0em,		% 오른쪽 여백  - 4em
						itemindent=0em, 		% 점 내민 거리 label sep 과 같은면 점위치 까지 내민다
						listparindent=0em}		% 본문 드려쓰기 간격
	
%			\begin{itemize}
	
		%	-------------------------------------------------------------------------------
		%		Label
		%	-------------------------------------------------------------------------------
			\renewcommand{\labelitemi}{$\bullet$}
			\renewcommand{\labelitemii}{$\bullet$}
%			\renewcommand{\labelitemii}{$\cdot$}
			\renewcommand{\labelitemiii}{$\diamond$}
			\renewcommand{\labelitemiv}{$\ast$}		
	
%			\renewcommand{\labelitemi}{$\blacksquare$}   	% 사각형 - 찬것
%			\renewcommand\labelitemii{$\square$}		% 사각형 - 빈것	
			






% ------------------------------------------------------------------------------
% Begin document (Content goes below)
% ------------------------------------------------------------------------------
	\begin{document}
	
			\dominitoc
			\doparttoc			




			\title{ 불교 반야선원 동안거 교재 }
			\author{김대희}
			\date{ 2020년 11월 }
			\maketitle


			\tableofcontents 		% 목차 출력
%			\listoffigures 			% 그림 목차 출력
			\cleardoublepage
			\listoftables 			% 표 목차 출력





		\mdfdefinestyle	{con_specification} {
						outerlinewidth		=1pt			,%
						innerlinewidth		=2pt			,%
						outerlinecolor		=blue!70!black	,%
						innerlinecolor		=white 			,%
						roundcorner			=4pt			,%
						skipabove			=1em 			,%
						skipbelow			=1em 			,%
						leftmargin			=0em			,%
						rightmargin			=0em			,%
						innertopmargin		=2em 			,%
						innerbottommargin 	=2em 			,%
						innerleftmargin		=1em 			,%
						innerrightmargin		=1em 			,%
						backgroundcolor		=gray!4			,%
						frametitlerule		=true 			,%
						frametitlerulecolor	=white			,%
						frametitlebackgroundcolor=black		,%
						frametitleaboveskip=1em 			,%
						frametitlebelowskip=1em 			,%
						frametitlefontcolor=white 			,%
						}



%	================================================================== Part			도서관 대출
	\addtocontents{toc}{\protect\newpage}
	\part[경허 선사 참선곡]{ 경허 선사 \\ 참선곡 }
	\noptcrule
	\parttoc				


% ----------------------------------------------------------------------------- 	2020.09
%										
% -----------------------------------------------------------------------------										
	\chapter{ 경허 선사 참선곡}


% ----------------------------------------------------------------------------- 	경허선사
%										
% -----------------------------------------------------------------------------									
	\section{ 경허 선사}






경허 스님은?

경허(1849∼1912) 스님은 9세 때 과천 청계사로 출가해 한학과 기초 불교경론을 배웠다. 이후 계룡산 동학사의 만화스님에게서 불교경론을 배우면서 제자백가를 섭렵했다. 1879년 옛 스승을 찾아가던 중 폭우를 만났으나 마침 돌림병의 유행으로 인가에 유숙할 수 없어 빗속에서 나무 아래 앉아 밤을 새다가 생사의 이치를 깨닫고 동학사로 돌아와 학인을 돌려보내고 조실방에 들어가 3개월 동안 면벽하여 크게 깨달았다. 부석사, 범어사 등에서 활동하며 많은 사람을 깨달음으로 이끌다 1912년 4월 갑산에서 입적했다.
저작권자 © 법보신문 무단전재 및 재배포 금지

% ----------------------------------------------------------------------------- 	경허선사
%										
% -----------------------------------------------------------------------------									
	\section{ 경허선사 성우(鏡虛禪師 惺牛) (1849∼1912) }

성은 송씨. 속명은 동욱(東旭), 법호는 경허(鏡虛).
\paragraph{}
9세 때 과천의 청계사(淸溪寺)로 출가하였다. 계허(桂虛)스님 밑에서 물긷고 나무하는 일로 5년을 보냈다. 그뒤 계룡산 동학사의 만화강백(萬化講伯)화상 밑에서 불교경론을 배웠으며, 9년 동안 그는 불교의 일대시교(一代時敎)뿐 아니라 <논어>·<맹자>·<시경>·<서경> 등의 유서(儒書)와 노장(老莊) 등의 제자백가를 모두 섭렵하였다.
\paragraph{}
1879년에 옛스승인 계허를 찾아 한양으로 향하던 중, 마을에 돌림병이 유행하여 집집마다 문을 굳게 닫고 있었다. 이제까지 생사불이(生死不二)의 이치를 문자 속에서만 터득하였음을 깨닫고 새로운 발심(發心)을 하였다. 이튿날, 동학사로 돌아와 학인들을 모두 돌려보낸 뒤 조실방(祖室房)에 들어가 용맹정진을 시작하였다. 창문 밑으로 주먹밥이 들어올 만큼의 구멍을 뚫어놓고, 한 손에는 칼을 쥐고, 목 밑에는 송곳을 꽂은 널판자를 놓아 졸음이 오면 송곳에 다치게 장치하여 잠을 자지않고 정진하였다. "소가 되더라도 콧구멍 없는 소가 되어야지."라는 말을 듣고 오도(悟道)하였다. 그뒤 천장암(天藏庵)으로 옮겨 깨달은 뒤에 수행인 보임(保任)을 하였다. 그때에도 얼굴에 탈을 만들어 쓰고, 송곳을 턱 밑에 받쳐놓고 오후수행(悟後修行)의 좌선을 계속하였다. 1886년 6년 동안의 보임공부(保任工夫)를 끝내고 옷과 탈바가지, 주장자 등을 모두 불태운 뒤 무애행(無碍行)을 하였고, 해인사 송광사 범어사 통도사등 제방의 선원에서 정진 하였고 만년에 천장암에서 최후의 법문을 한 뒤 사찰을 떠나 갑산(甲山)·강계(江界) 등지에서 머리를 기르고 유관(儒冠)을 쓴 모습으로 살았으며, 박난주(朴蘭州) 라고 개명하였다. 그곳에서 서당의 훈장이 되어 아이들을 가르치다가, 1912년 4월 25일 새벽에 입적하였다. 나이 64세, 법랍 56세이다. 저서에는 <경허집>이 있다.


% ----------------------------------------------------------------------------- 	경허선사 참선곡
%										
% -----------------------------------------------------------------------------									
	\section{ 경허 선사 참선곡 }


홀연히 생각하니 도시몽중이로다/ 천만고 영웅호걸 북망산 무덤이요/ 부귀문장 쓸데없다 황천객을 면할소냐/ 오호라 나의몸이 풀끝에 이슬이요/ 바람속에 등불이라 삼계대사 부처님이/ 정령히 이르사대 마음깨쳐 성불하여/ 생사윤회 영단하고 불생불멸 저국토에/ 상낙아정 무위도를 사람마다 다할줄로/ 팔만장교 유전이라 사람되어 못닦으면/ 다시공부 어려우니 나도어서 닦아보세/ 닦는길을 말하려면 허다히 많건마는 대강추려 적어보세.\\

앉고서고 보고듣고 착의긱반 대인접화 일체처 일체시에/ 소소영영 지각하는 이것이 무엇인고/ 몸둥이는 송장이요 망상번뇌 본공하고/ 천진면목 나의부처 보고듣고 앉고눕고/ 잠도자고 일도하고 눈한번 깜짝할세/ 천리만리 다녀오고 허다한 신통묘용/ 분명한 나의마음 어떻게 생겼는고/ 의심하고 의심하되 고양이가 쥐잡듯이/ 주린사람 밥찾듯이 목마른때 물찾듯이/ 육칠십 늙은과부 외자식을 잃은후에/ 자식생각 간절하듯 생각생각 잊지말고/ 깊이궁구 하여가되 일념만년 되게하야/ 폐침망찬 할지경에 대오하기 가깝도다.\\

헛튼소리 우시개로 이날저날 헛보내고/ 늙는줄을 망각하니 무슨공부 하여볼까/ 죽을제 고통중에 후회한들 무엇하리/ 사지백절 오려내고 머릿골을 쪼개낸듯/ 오장육부 타는중에 앞길이 캄캄하니/ 한심참혹 내노릇이 이럴줄을 누가알꼬/ 저지옥과 저축생의 나의신세 참혹하다/ 백천만겁 차타하여 다시인신 망연하다.\\

참선잘한 저도인은 서서죽고 앉아죽고/ 앓도않고 선세하며 오래살고 곧죽기를/ 마음대로 자재하며 항하사수 신통묘용/ 임의쾌락 소요하니 아무쪼록 이세상에/ 눈코를 쥐어뜯고 부지런히 하여보세/ 오늘내일 가는것이 죽을날에 당도하니/ 포주간에 가는소가 자욱자욱 사지로세/ 예전사람 참선할제 잠오는것 성화하여/ 송곳으로 찔렀거늘 나는어이 방일하며/ 예전사람 참선할제 하루해가 가게되면/ 다리뻗고 울었거늘 나는어이 방일한고

무명업식 독한술에 혼혼불각 지내다니/ 오호라 슬프도다 타일러도 아니듣고/ 꾸짖어도 조심않고 심상히 지내가니/ 혼미한 이마음을 어이하야 인도할꼬/ 쓸데없는 탐심진심 공연히 일으키고/ 쓸데없는 허다분별 날마다 분요하니. 우습도다 나의지혜 누구를 한탄할꼬/ 지각없는 저나비가 불빛을 탐하여서/ 제죽을줄 모르도다 내마음을 못닦으면/ 여간계행 소분복덕 도무지 허사로세/ 오호라 한심하다 이글을 자세보아/ 하루도 열두때며 밤으로도 조금자고/ 부지런히 공부하소 이노래를 깊이믿어/ 책상위에 피여놓고 시시때때 경책하소/ 할말을 다하려면 해묵서이 부진이라/ 이만적고 그치오니 부디부디 깊이아소/ 다시할말 있아오니 돌장승이 아기나면/ 그때다시 말할테요.


% ----------------------------------------------------------------------------- 	경허선사 참선곡
%										
% -----------------------------------------------------------------------------									
	\section{ 경허 선사 참선곡 }



홀연히 생각하니 도시몽중(都是夢中)이로다
천만고 영웅 호걸 북망산(北邙山) 무덤이요.
부귀문장 쓸데 없다. 황천객(黃泉客)을 면할소냐?
오호라, 나의 몸이 풀 끝에 이슬이요 바람속에 등불이라.
삼계대사(三界大師)부처님이 정녕히 이르사대
마음 깨쳐 성불하여 생사윤회(生死輪回) 영단(永斷)하고
불생불멸(不生不滅) 저 국토에 상락아정(常樂我淨) 무위도(無爲道)를
사람마다 다 할 줄로 팔만장교(八萬藏敎) 유전이라.
사람되어 못닦으면 다시 공부 어려우니 나도 어서
닦아보세. 닦는 길을 말 하랴면 허다히 많건마는
대강추려 적어보세 안고 서고 보고 듣고 착의긱반(着衣喫飯)
대인접화(對人接話) 일체처 일체시에 소소영영(昭昭靈靈)
지각(知覺)하는 이것이 무엇인고? 몸둥이는 송장이요
망상번뇌 본공(本空)하고 천진면목(天眞面目) 나의부처
보고 듣고 앉고 서고, 잠도자고 일도하고
눈 한번 깜짝할제 천리 만리 다녀오고 허다한
신통묘용(神通妙用) 분명한 나의마음 어떻게 생겼는고?
의심하고 의심하되 고양이가 쥐 잡듯이 주린사람
밥 찾듯이 목 마른데 물 찾듯이 육 칠십 늙은과부
외자식을 잃은후에 자식생각 간절하듯 생각 생각
잊지말고 깊히궁구 하여가되 일념만년(一念萬年)
되게하여 폐침망찬(廢寢忘饌) 할 지경에 대오(大悟)
하기 가깝도다. 홀연히 깨달으면 본래생긴 나의부처
천진면목(天眞面目) 절묘(絶妙)하다.
아미타불 이 아니며 석가여래 이 아닌가?
젊도않고 늙도않고 크도않고 적도않고 본래생긴
자기영광(自己靈光) 개천개지(蓋天蓋地)
이러하고 열반진락(涅槃眞樂) 가이 없다.
지옥천당 본공(本空)하고 생사윤회(生死輪回) 본래없다.
선지식(善知識)을 찾아가서 요연(了然)히 인가(印可)맞아
다시 의심 없앤 후에 세상만사 망각(忘却)하고
수연방광(隨緣放曠) 지내가되 빈배 같이 떠놀면서
유연중생(有緣衆生)제도하면 보불은덕(報佛恩德) 이 아닌가?
일체계행 지켜가면 천상인간 복수(福壽)하고 대원력을
발하여서 항수불학(恒隨佛學) 생각하고, 동체대비(同體大悲)
마음먹어 빈병걸인(貧病乞人) 괄시(恝視)말고,
오온색신(五蘊色身) 생각하되 거품같이 관(觀)을 하고,
바깥으로 역순경계(逆順境界) 부동(不動)한 이 마음을
태산(泰山)같이 써 나가세.
허튼 소리 우시게로 이날 저날 다 보내고 늙을 줄을 망각하니
무슨 공부 하여볼까? 죽을 때 고통중에 후회한들 무엇하리.
사지백절(四肢百節) 오려내고 머릿골을 쪼개는 듯
오장육부 (五臟六腑) 타는 중에 앞길이 캄캄하니,
한심참혹(寒心慘酷) 내 노릇이 이럴 줄을 뉘가 알꼬.
저지옥과 저 축생(畜生)에 나의 신세(身勢) 참혹하다.
백천만겁(百千萬劫) 차타(蹉 )하여 다시 인신 망연(茫然)하다
참선잘한 저 도인은 서서죽고 앉아죽고 앓도않고 선세하며
오래살고 곧 죽기를 마음대로 자재하며, 항하사수(恒河沙數)
신통묘용(神通妙用) 임의쾌락(任意快樂) 소요(逍遙)하니,
아무쪼록 이 세상에 눈 코를 쥐어 뜯고 부지런히 하여보세.
오늘 내일 가는 것이 죽을 날에 당도(當到)하니 포주간에
가는 소가 자욱자욱 사지로세.
예전사람 참선(參禪)할제 마디 그늘 아꼈거는
나는 어이 방일(放逸)하며,
예전사람 참선할제 잠오는 것 성화하여 송곳으로 찔렀거늘
나는 어이 방일하며,
예전 사람 참선할제 하루 해가 가게 되면 다리뻗고 울었거는
나는 어이 방일한고.
무명업식(無明業識) 독한 술에 혼혼불각(昏昏不覺) 지내가니,
오호(嗚呼)라 슬프도다.
타일러도 아니 듣고 꾸짖어도 조심(操心)않고 심상(尋常)히
지내가니 혼미(昏迷)한 이마음을 어이하여 인도(引導)할꼬.
쓸데 없는 탐심진심(貪心嗔心) 공연(空然)히 일으키고 쓸데없는
허다분별(許多分別) 날마다 분요(紛擾)하니 우습도다 나의 지혜
누구를 한탄할고?
지각없는 저 나비가 불빛을 탐하여서 제 죽을줄 모르도다.
내마음을 못 닦으면 여간 계행 (如干戒行) 소분복덕(小分福德)
도무지 허사(虛事)로세.
오호라 한심하다
이 글을 자세(仔細)보아 하루도 열두 때며 밤으로도
조금 자고 부지런히 공부하소.
할 말을 다하려면 해북서이부진(海墨書而不盡)이라.
이만 적고 끝내 오니 부디 부디깊이 아소.
다시 할 말 있아오니 돌 장승이 아이나면 그때에 말하리라.



% ----------------------------------------------------------------------------- 	경허선사 참선곡
%										
% -----------------------------------------------------------------------------									
	\section{ 경허선사 참선곡(參禪曲)  한문해설 }



홀연히 생각하니 만사(萬事) 도시몽중(都是夢中)이라 천만고 영웅호걸 북망산(北邙山) 무덤이요 부귀문장 쓸데없다 황천객(黃泉客)을 면할소냐 오호라 나의몸이 풀끝에 이슬이요 바람속에 등불이라 삼계대사 부처님이 정녕(丁寧)히 이르사대 마음깨쳐 성불하여 생사윤회 영단(永斷)하고 불생불멸 저국토에 상락아정(常樂我淨) 무위도(無爲道)를 사람마다 다할줄을 팔만장교(八萬藏敎) 유전(遺傳)이라

*도시몽중(都是夢中): 아득한 꿈속
*북망산(北邙山): 중국 낙양 북쪽에 있는 산으로 왕후나 공경대부등 사람이 죽으면 이곳에 와서 묻혔다고 한다.
*황천객(黃泉客): 저승으로 가는 나그네
*정녕(丁寧)히 : 간곡하게
*영단(永斷): 영원히 남김없이 끊어버리는 것
*상락아정(常樂我淨): 열반의 세계는 절대 영원하고 즐겁고 자재(自在)한  참된 자아가 확립되어 있으며 청정함을 이른다
* 무위도(無爲道): 무심의 경지를 무위도라 한다. 무위심내기비심 무주상보시 기복덕불가사량
* 팔만장교(八萬藏敎) : 팔만사천 부처님 법문
* 유전(遺傳) : 대를 이어 전해지는 것

 사람되어 못닦으면 다시공부 어려우니 나도어서 닦아보세 닦는길을 말하려면 허다히 많건마는 대강추려 적어보세 앉고서고 보고듣고 착의끽반(着衣喫飯) 대인접화(對人接話) 일체처 일체시에 소소영영(昭昭靈靈) 지각하는 이것이 무엇인고 몸뚱이는 송장이요 망상번뇌 본공(本空)하고 천진면목(天眞面目) 나의부처 보고듣고 앉고눕고 잠도자고 일도하고 눈한번 깜짝할제 천리만리 다녀오고 허다한 신통묘용 분명한 나의마음 어떻게 생겼는고 의심하고 의심하되 고양이가 쥐잡듯이 주린사람 밥찾듯이 목마른데 물찾듯이 육칠십 늙은과부 외자식을 잃은후에 자식생각 간절하듯 생각생각 잊지말고 깊이궁구 하여가되 일념만년(一念萬年) 되게하여 폐침망찬(廢寢忘饌) 할지경에 대오하기 가깝도다 홀연히 깨달으면 본래생긴 나의부처 천진면목 절묘하다 아미타불 이아니며 석가여래 이아닌가 젊도않고 늙도않고 크도않고 작도않고 본래생긴 자기영광 개천개지(盖天盖地) 이러하고 열반진락(涅般眞樂) 가이없다 지옥천당 본공하고 생사윤회 본래없다 선지식을 찾아가서 요연(了然)히 인가마쳐 다시의심 없앤후에 세상만사 망각하고 수연방광(隨緣放曠) 지내가되

*착의끽반(着衣喫飯) : 옷입고 밥먹는것
*대인접화(對人接話) : 사람만나고 애기하는
*소소영영(昭昭靈靈) : 한없이 밝고 신령스럽다
*본공(本空) : 본래 비어있는
*천진면목(天眞面目) : 본래의 참모습
*일념만년(一念萬年) : 한생각이 만년가듯
*폐침망찬(廢寢忘饌) : 몰두하여 침식을 잊다
*개천개지(盖天盖地) : 하늘과 땅을 덮어 가린다는 뜻으로, 중생(衆生)이 본래 갖추고 있는 마음의 빛이 하늘과 땅에 가득 참을 이르는 말
*열반진락(涅般眞樂) : 깨달음의 참다운 즐거움
*요연(了然) : 마침내 분명하다
*수연방광(隨緣放曠) : 언행에서 거리낌이 없는 도인의 경지

빈배같이 떠돌면서 유연중생(有緣衆生) 제도하면 보불은덕(報佛恩德) 이아닌가 일체계행 지켜가면 천상인간 복수(福壽)하고 대원력을 발하여서 항수불학(恒隨佛學) 생각하고 동체대비(同體大悲) 마음먹어 빈병걸인(貧病乞人) 괄시말고 오온색신(五蘊色身) 생각하되 거품같이 관을하고 바깥으로 역순경계(逆順境界) 몽중으로 관찰하여 해태심(懈怠心)을 내지말고 허령(虛靈)한 나의마음 허공과 같은줄로 진실히 생각하여 팔풍오욕(八風五慾)

 

*유연중생(有緣衆生) : 인연있는 중생
*보불은덕(報佛恩德) : 부처님은혜에 보답하는것
*복수(福壽) : 복을 받음
*항수불학(恒隨佛學) : 항상 부처님을 따라 배움
*동체대비(同體大悲) : 중생의 고통을 나의 고통으로 여기는 불보살의 자비심
*빈병걸인(貧病乞人) : 가난한사람과 병에걸린사람과 얻어먹는 거지
*오온색신(五蘊色身) : 물질적인 것을 의미하는 색, 감각의 수, 인식 작용의 상, 의지 작용의 행, 마음 작용의 식 으로 구성된 몸
*역순경계(逆順境界) : 좋고 나쁨을 이으키는 마음의 경계
*해태심(懈怠心) : 게으르고 나태한 마음
*허령(虛靈) : 잡된 생각이 없이 신령스런 마음
*팔풍(八風) :

           이利---남이 나에게 이롭게 하는 것(흔들리지 말라)
           쇠衰---내외 형편이 쇠잔해지는것
           훼毁---남이 나를 헐뜻고 비방할 때나
           예譽---모든 일이 내 뜻대로 되는것
           칭稱---나를 칭찬할 때나
           기譏---남이 나를 제 맘대로 희롱할 때나
           고苦---고생스러울 때나 
           락樂---편안하고 즐거운 때나
 *오욕(五慾): 식욕(食慾)ㆍ색욕(色慾)ㆍ재욕(財慾)ㆍ명예욕ㆍ수면욕

  

일체경계 부동한 이마음을 태산같이 써나가세 허튼소리 우스개로 이날저날 헛보내고 늙은줄을 망각하니 무슨공부 하여볼까 죽을제 고통중에 후회한들 무엇하리 사지백절(四肢百節) 오려내고 머리골을 쪼개낸듯 오장육부 타는중에 앞길이 캄캄하니 한심참혹(寒心慘酷) 내노릇이 이럴줄을 누가알꼬 저지옥과 저축생의 나의신세 참혹하다 백천만겁 차타(蹉打)하여 다시 인신(人身) 망연(茫然)하다 참선 잘 한 저도인은 서서죽고 앉아죽고 앓도 않고 선세(蟬兌)하며

 

*사지백절(四肢百節) : 팔다리와 온몸의 관절
*한심참혹(寒心慘酷) : 마음이 떨리고 비참하고 끔찍함
*차타(蹉打): 일을 이루지 못하고 나이가 들어감
*망연(茫然) :  아득하여 어렵다         
*선세(蟬兌) : 매미가 몸바꾸듯 새로이 몸을 얻음

 

오래살고 곧죽기를 마음데로 자재하며 항하사수(恒河沙數) 신통묘용(神通妙用) 임의괘락(任意快樂) 소요하니 아무쪼록 이세상에 눈코를 쥐어뜯고 부지런히
하여보세 오늘내일 가는것이 죽을날에 당도하니 푸주간에 가는소가
자욱자욱 사지로세 예전사람 참선할제 마디그늘 아꼈거늘 나는어이
방일하며 예전사람 참선할제 잠오는것 성화하여 송곳으로 찔럿거늘
나는어이 방일하며 예전사람 참선할제 하루해가 가게되면 다리뻗고 울었거늘 나는어이 방일한고 무명업식(無明業識) 독한술에 혼혼불각(昏昏不覺)

 

*항하사수(恒河沙數) : 인도 갠지스강(항하)의 모래알 수
*신통묘용(神通妙用) : 신족통(神足通), 천안통(天眼通), 천이통(天耳通), 타심통(他心通), 숙명통(宿命通), 누진통(漏盡通)의 육신통의 묘한 사용
*임의괘락(任意快樂) : 자신의 의지대로 즐거움을 누림
*무명업식(無明業識) : 어리석음으로 가려져 밝은 진리를 얻지못하고 계속 어리석음의 과보를 만드는 인식
* 혼혼불각(昏昏不覺) : 정신이 가물가물하고 희미하여 진리를 깨닫지 못함

 

지내가니 오호라 슬프도다 타일러도 아니듣고 꾸짖어도 조심않고 심상히 지내가니 혼미한 이마음을 어이하야 인도할고 쓸데없는 탐심진심(貪心嗔心) 공연히 일으키고 쓸데없는 허다분별(許多分別) 날마다 분요(紛擾)하니 우습도다 나의지혜 누구를 한탄할고 지각없는 저나비가 불빛을 탐하여서 제죽을줄모르도다 내마음을 못닦으면 여간계행(如間戒行) 소분복덕(少分福德) 도무지 허사로세 오호라 한심하다 이글을 자세보아 하루도 열두때며 밤으로도 조금자고 부지런히 공부하소 이노래를 깊이믿어 책상위에 펴어놓고 시시때때 경책하소 할말을 다하려면 해묵서이부진(海墨書而不盡)이라 이만적고 그치오니 부디부디 깊이아소 다시한말 있사오니 돌장승이 아기나면 그때에 말하리라.

 


*탐심진심(貪心嗔心) : 탐내고 성내는 마음
*허다분별(許多分別) : 자신의 생각만을 기준으로 하여 대상을 판단하는 어리석은 의식작용을 매우 많이함
*분요(紛擾) : 먼지가 날리듯이 어지럽다.
*여간계행(如間戒行) : 계율을 잘 지킨다.
*소분복덕(少分福德) : 적은 복덕을 짓는다고 하여도
*해묵서이부진(海墨書而不盡) : 바다물로 써도 다함이 없다.   





% ----------------------------------------------------------------------------- 	경허의 세달
%										
% -----------------------------------------------------------------------------									
	\section{ 경허의 세 달}

경허 선사의 수제자로 흔히 '삼월(三月)'로 불리는 혜월(慧月, 1861년 - 1937년), 수월(水月, 1855년 - 1928년)·만공(滿空, 1871년 - 1946년) 선사가 있다. 경허는 '만공은 복이 많아 대중을 많이 거느릴 테고, 정진력은 수월을 능가할 자가 없고, 지혜는 혜월을 당할 자가 없다'고 했다. 삼월인 제자들도 모두 깨달아 부처가 되었다. 이들 역시 근현대 한국 불교계를 대표하는 선승들이다.

1904년 7월 15일, 만공스님에게 전법게를 주고서, 천장암을 떠났다.

법자 만공에게 주다
수산 월면에게 글자 없는 도장을 부쳐 주고 주장자를 잡아 한 번 치고 이르기를 "다만 이 말소리가 이것이다. 라고 하였으니 또 말해 봐라. 이 무슨 도리인가?"
또 한 번 치고 이르기를 "한 번 웃고는 아지 못커라, 낙처가 어디인가. 안면도의 봄물이 푸르기를 쪽과 같도다." 하고 주장자를 던지고 흐음하고 내려오다.
현재, '북송담 남진제'의 두 큰스님의 경우에, 송담스님은 경허(75대)-만공(76대)-전강(77대)-송담(78대)의 계보이고, 진제스님은 경허(75대)-혜월(76대)-운봉(77대)-향곡(78대)-진제(79대)의 계보이다.

연표

% ----------------------------------------------------------------------------- 	만공 스님
%										
% -----------------------------------------------------------------------------										
	\chapter{ 만공 스님}


% ----------------------------------------------------------------------------- 	경허선사 참선곡
%										
% -----------------------------------------------------------------------------									
	\section{ 만공스님의 탄금법곡 }



다시듣는 큰스님 법문 2 만공스님
 승인 2002.01.19 13:04 호수 152 댓글 0페이스북

만공 월면스님
“머무르되 머무른바 없다” 달을 물에 비춰 집을 삼는지하 머무를 바 없다 만공(滿空1871~1946)스님은 경허스님의 법을 계승하고 선지종풍(禪旨宗風)을 진작시킨 고승이다. 40여년간 덕숭산에 주석하며 진리의 법을 전한 만공스님은 한국불교 선교양종의 초대 교정으로 추대됐으며, 일제의 식민불교정책에 반대하는 등 한국 전통불교의 맥을 계승하기위한 노력을 아끼지 않았다.만공스님이 1937년 금선동 초당에서 휘영청 밝은 달이 뜬 고요한 밤에 거문고와 함께한 ‘탄금법곡(彈琴法曲)이란 제목의 법문이다. “일탄운시심곡(一彈云是甚 曲), 시채현곡야(是體玄曲也) / 일탄운시심곡(一彈云是甚 曲), 시구현곡야(是句玄曲也) / 일탄운시심곡(一彈云是甚 曲), 시현현곡야(是玄玄曲也) / 일탄운시심곡(一彈云是甚 曲), 시석여심중겁외곡야(是石女心中劫外曲也)” 우리말로 옮기면 이렇다. “한번 퉁기고 이르노니 이는 무슨 곡조인가. 이는 체의 현현한 곡이로다. / 한번 퉁기고 이르노니 이는 무슨 곡조인가. 이는 일구의 현현한 곡이로다. / 한번 퉁기고 이르노니 이는 무슨 곡조인가. 이는 현현하고 현현한 곡이로다. / 한번 퉁기고 이르노니 이는 무슨 곡조인가. 이는 돌장승의 마음 가운데 겁 밖의 곡이로다. 아하 ” 만공스님은 ‘윤회의 자취(輪廻之跡)’를 이렇게 설법했다. 스님의 법문을 모아 놓은 〈만공법어〉에 실린 내용이다. “나고 죽음에 윤회의 자취가 다함이 없고, 고요하고 뚜렷하매 참으로 비추는 이 기틀이 매(昧)하지 않도다. 구름은 산을 의지하여 아비를 삼는지라, 이 낱 가운데 공덕으로써 공덕에 나아감이여, 달은 물에 비추어 집을 삼는지라, 곧 머무르되 머무른 바가 없음이니라. 보고 듣고 깨닫고 아는 것을 여의고깊이 지혜가 있으니, 이는 분별의 마음이 아니다. 땅과 물과 불과 바람을 여의고 특별히 몸이 있으니, 곧 화합의 모습이 아니로다. 그러므로, 이르되 사대(四大)의 성품이 스스로 회복하여 아들이 그의 어미를 얻은 것과 같도다. 여러 선덕은 어떻게 생각하는가. 이러한 행리(行履)를 증득해야 서로 응해 갈 수 있으리라. 도리어 알겠는가. 서리 찬 하늘에 달은 지고 밤이 깊었는데, 누가 맑은 못 찬 그림자를 비출고.”여기서 ‘매(昧)’는 어두움을 나타낸다. 만공스님은 후학들에게‘나를 찾는법’이란 가르침을 통해 “공부를 잘하고 못하는 문제 보다도 이 공부밖에 할 일이 없다는 결정적 신심(信心)부터 세워야 한다”면서 “사람을 대할때는 자비심으로 대하여야 하지만, 공부를 위해서는 극악극독심(極惡極毒心)이 아니면 팔만사천 번뇌마(煩惱魔)를 쳐부수지 못하나니라”고 강조했다. 스님은 1946년 10월20일 목욕 단좌(端坐)한 후 거울에 비친 당신의 모습을 보고 “자네와 내가 이제 이별할 인연이 다 되었네 그려”하고 입적에 들었다. 정리=李成洙기자 soolee@buddhism.or.kr
저작권자 © 불교신문 무단전재 및 재배포 금지
200701 불교신문후원 기사뷰



% ----------------------------------------------------------------------------- 	2020.09
%										
% -----------------------------------------------------------------------------										
	\chapter{ 유식의 이해 }
	\chapter{ 식이 전변한 세상 }
	\chapter{ 첫번째 능변식,제8식 }
	\chapter{ 두번째 능변식,제7식 }
	\chapter{ 세번째 능변식,제6식 }
	\chapter{ 세상의 모습 (삼성과 삼무성) }
	\chapter{ 대승보살의 길 }
	\chapter{ 글을 마치며}




%	================================================================== Part			2019
	\addtocontents{toc}{\protect\newpage}
	\part{ 2019년  }
	\noptcrule
	\parttoc				



%	================================================================== Part		지은이
	\addtocontents{toc}{\protect\newpage}
	\part{ 지은이  }
	\noptcrule
	\parttoc				





% ------------------------------------------------------------------------------
% End document
% ------------------------------------------------------------------------------
\end{document}


	\href{https://www.youtube.com/watch?v=SpqKCQZQBcc}{태양경배자세A}
	\href{https://www.youtube.com/watch?v=CL3czAIUDFY}{태양경배자세A}


https://docs.google.com/spreadsheets/d/1-wRuFU1OReWrtxkhaw9uh5mxouNYRP8YFgykMh2G_8c/edit#gid=0
+

https://seoyeongcokr-my.sharepoint.com/:f:/g/personal/02017_seoyoungeng_com/Ev8nnOI89D1LnYu90SGaVj0BTuckQ46vQe1HiVv-R4qeqQ?e=S3iAHi