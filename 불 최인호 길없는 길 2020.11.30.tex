%	-------------------------------------------------------------------------------
% 
%		2020년 11월 
%		30일 일요일 
%		첫작업
%
%
%
%
%
%
%
%	-------------------------------------------------------------------------------

%	\documentclass[12pt, a3paper, oneside]{book}
	\documentclass[12pt, a4paper, oneside]{book}
%	\documentclass[12pt, a4paper, landscape, oneside]{book}

		% --------------------------------- 페이지 스타일 지정
		\usepackage{geometry}
%		\geometry{landscape=true	}
		\geometry{top 			=10em}
		\geometry{bottom			=10em}
		\geometry{left			=8em}
		\geometry{right			=8em}
		\geometry{headheight		=4em} % 머리말 설치 높이
		\geometry{headsep		=2em} % 머리말의 본문과의 띠우기 크기
		\geometry{footskip		=4em} % 꼬리말의 본문과의 띠우기 크기
% 		\geometry{showframe}
	
%		paperwidth 	= left + width + right (1)
%		paperheight 	= top + height + bottom (2)
%		width 		= textwidth (+ marginparsep + marginparwidth) (3)
%		height 		= textheight (+ headheight + headsep + footskip) (4)	



		%	===================================================================
		%	package
		%	===================================================================
%			\usepackage[hangul]{kotex}				% 한글 사용
			\usepackage{kotex}					% 한글 사용
			\usepackage[unicode]{hyperref}			% 한글 하이퍼링크 사용

		% ------------------------------ 수학 수식
			\usepackage{amssymb,amsfonts,amsmath}	% 수학 수식 사용
			\usepackage{mathtools}				% amsmath 확장판

			\usepackage{scrextend}				% 
		

		% ------------------------------ LIST
			\usepackage{enumerate}			%
			\usepackage{enumitem}			%
			\usepackage{tablists}				%	수학문제의 보기 등을 표현하는데 사용
										%	tabenum


		% ------------------------------ table 
			\usepackage{longtable}			%
			\usepackage{tabularx}			%
			\usepackage{tabu}				%




		% ------------------------------ 
			\usepackage{setspace}			%
			\usepackage{booktabs}		% table
			\usepackage{color}			%
			\usepackage{multirow}			%
			\usepackage{boxedminipage}	% 미니 페이지
			\usepackage[pdftex]{graphicx}	% 그림 사용
			\usepackage[final]{pdfpages}		% pdf 사용
			\usepackage{framed}			% pdf 사용

			
			\usepackage{fix-cm}	
			\usepackage[english]{babel}
	
		%	=======================================================================================
		% 	tikz package
		% 	
		% 	--------------------------------- 	
			\usepackage{tikz}%
			\usetikzlibrary{arrows,positioning,shapes}
			\usetikzlibrary{mindmap}			
			

		% --------------------------------- 	page
			\usepackage{afterpage}		% 다음페이지가 나온면 어떻게 하라는 명령 정의 패키지
%			\usepackage{fullpage}			% 잘못 사용하면 다 흐트러짐 주의해서 사용
%			\usepackage{pdflscape}		% 
			\usepackage{lscape}			%	 


			\usepackage{blindtext}
	
		% --------------------------------- font 사용
			\usepackage{pifont}				%
			\usepackage{textcomp}
			\usepackage{gensymb}
			\usepackage{marvosym}



		% Package --------------------------------- 

			\usepackage{tablists}				%


		% Package --------------------------------- 
			\usepackage[framemethod=TikZ]{mdframed}				% md framed package
			\usepackage{smartdiagram}								% smart diagram package



		% Package ---------------------------------    연습문제 

			\usepackage{exsheets}				%

			\SetupExSheets{solution/print=true}
			\SetupExSheets{question/type=exam}
			\SetupExSheets[points]{name=point,name-plural=points}


		% --------------------------------- 페이지 스타일 지정

		\usepackage[Sonny]		{fncychap}

			\makeatletter
			\ChNameVar	{\Large\bf}
			\ChNumVar	{\Huge\bf}
			\ChTitleVar		{\Large\bf}
			\ChRuleWidth	{0.5pt}
			\makeatother

%		\usepackage[Lenny]		{fncychap}
%		\usepackage[Glenn]		{fncychap}
%		\usepackage[Conny]		{fncychap}
%		\usepackage[Rejne]		{fncychap}
%		\usepackage[Bjarne]	{fncychap}
%		\usepackage[Bjornstrup]{fncychap}

		\usepackage{fancyhdr}
		\pagestyle{fancy}
		\fancyhead{} % clear all fields
		\fancyhead[LO]{\footnotesize \leftmark}
		\fancyhead[RE]{\footnotesize \leftmark}
		\fancyfoot{} % clear all fields
		\fancyfoot[LE,RO]{\large \thepage}
		%\fancyfoot[CO,CE]{\empty}
		\renewcommand{\headrulewidth}{1.0pt}
		\renewcommand{\footrulewidth}{0.4pt}
	
	
	
		%	--------------------------------------------------------------------------------------- 
		% 	tritlesec package
		% 	
		% 	
		% 	------------------------------------------------------------------ section 스타일 지정
	
			\usepackage{titlesec}
		
		% 	----------------------------------------------------------------- section 글자 모양 설정
			\titleformat*{\section}					{\large\bfseries}
			\titleformat*{\subsection}				{\normalsize\bfseries}
			\titleformat*{\subsubsection}			{\normalsize\bfseries}
			\titleformat*{\paragraph}				{\normalsize\bfseries}
			\titleformat*{\subparagraph}				{\normalsize\bfseries}
	
		% 	----------------------------------------------------------------- section 번호 설정
			\renewcommand{\thepart}				{\arabic{part}.}
			\renewcommand{\thesection}				{\arabic{section}.}
			\renewcommand{\thesubsection}			{\thesection\arabic{subsection}.}
			\renewcommand{\thesubsubsection}		{\thesubsection\arabic{subsubsection}}
			\renewcommand\theparagraph 			{$\blacksquare$ \hspace{3pt}}

		% 	----------------------------------------------------------------- section 페이지 나누기 설정
			\let\stdsection\section
			\renewcommand\section{\newpage\stdsection}



		%	--------------------------------------------------------------------------------------- 
		% 	\titlespacing*{commandi} {left} {before-sep} {after-sep} [right-sep]		
		% 	left
		%	before-sep		:  수직 전 간격
		% 	after-sep	 	:  수직으로 후 간격
		%	right-sep

			\titlespacing*{\section} 			{0pt}{1.0em}{1.0em}
			\titlespacing*{\subsection}	  		{0ex}{1.0em}{1.0em}
			\titlespacing*{\subsubsection}		{0ex}{1.0em}{1.0em}
			\titlespacing*{\paragraph}			{0em}{1.5em}{1.0em}
			\titlespacing*{\subparagraph}		{4em}{1.0em}{1.0em}
	
		%	\titlespacing*{\section} 			{0pt}{0.0\baselineskip}{0.0\baselineskip}
		%	\titlespacing*{\subsection}	  		{0ex}{0.0\baselineskip}{0.0\baselineskip}
		%	\titlespacing*{\subsubsection}		{6ex}{0.0\baselineskip}{0.0\baselineskip}
		%	\titlespacing*{\paragraph}			{6pt}{0.0\baselineskip}{0.0\baselineskip}
	

		% --------------------------------- recommend		섹션별 페이지 상단 여백
		\newcommand{\SectionMargin}				{\newpage  \null \vskip 2cm}
		\newcommand{\SubSectionMargin}			{\newpage  \null \vskip 2cm}
		\newcommand{\SubSubSectionMargin}		{\newpage  \null \vskip 2cm}


		%	--------------------------------------------------------------------------------------- 
		% 	toc 설정  - table of contents
		% 	
		% 	
		% 	----------------------------------------------------------------  문서 기본 사항 설정
			\setcounter{secnumdepth}{4} 		% 문단 번호 깊이
			\setcounter{tocdepth}{2} 			% 문단 번호 깊이 - 목차 출력시 출력 범위

			\setlength{\parindent}{0cm} 		% 문서 들여 쓰기를 하지 않는다.


		%	--------------------------------------------------------------------------------------- 
		% 	mini toc 설정
		% 	
		% 	
		% 	--------------------------------------------------------- 장의 목차  minitoc package
			\usepackage{minitoc}

			\setcounter{minitocdepth}{1}    	%  Show until subsubsections in minitoc
%			\setlength{\mtcindent}{12pt} 	% default 24pt
			\setlength{\mtcindent}{24pt} 	% default 24pt

		% 	--------------------------------------------------------- part toc
		%	\setcounter{parttocdepth}{2} 	%  default
			\setcounter{parttocdepth}{0}
		%	\setlength{\ptcindent}{0em}		%  default  목차 내용 들여 쓰기
			\setlength{\ptcindent}{0em}         


		% 	--------------------------------------------------------- section toc

			\renewcommand{\ptcfont}{\normalsize\rm} 		%  default
			\renewcommand{\ptcCfont}{\normalsize\bf} 	%  default
			\renewcommand{\ptcSfont}{\normalsize\rm} 	%  default


		%	=======================================================================================
		% 	tocloft package
		% 	
		% 	------------------------------------------ 목차의 목차 번호와 목차 사이의 간격 조정
			\usepackage{tocloft}

		% 	------------------------------------------ 목차의 내어쓰기 즉 왼쪽 마진 설정
			\setlength{\cftsecindent}{2em}			%  section

		% 	------------------------------------------ 목차의 목차 번호와 목차 사이의 간격 조정
			\setlength{\cftsecnumwidth}{2em}		%  section





		%	=======================================================================================
		% 	flowchart  package
		% 	
		% 	------------------------------------------ 목차의 목차 번호와 목차 사이의 간격 조정
			\usepackage{flowchart}
			\usetikzlibrary{arrows}


		%	=======================================================================================
		% 		makeindex package
		% 	
		% 	------------------------------------------ 목차의 목차 번호와 목차 사이의 간격 조정
%			\usepackage{makeindex}
%			\usepackage{makeidy}


		%	=======================================================================================
		% 		각주와 미주
		% 	

		\usepackage{endnotes} %미주 사용


		%	=======================================================================================
		% 	줄 간격 설정
		% 	
		% 	
		% 	--------------------------------- 	줄간격 설정
			\doublespace
%			\onehalfspace
%			\singlespace
		
		

	% 	============================================================================== itemi Global setting

	
		%	-------------------------------------------------------------------------------
		%		Vertical spacing
		%	-------------------------------------------------------------------------------
			\setlist[itemize]{topsep=0.0em}			% 상단의 여유치
			\setlist[itemize]{partopsep=0.0em}			% 
			\setlist[itemize]{parsep=0.0em}			% 
%			\setlist[itemize]{itemsep=0.0em}			% 
			\setlist[itemize]{noitemsep}				% 
			
		%	-------------------------------------------------------------------------------
		%		Horizontal spacing
		%	-------------------------------------------------------------------------------
			\setlist[itemize]{labelwidth=1em}			%  라벨의 표시 폭
			\setlist[itemize]{leftmargin=8em}			%  본문 까지의 왼쪽 여백  - 4em
			\setlist[itemize]{labelsep=3em} 			%  본문에서 라벨까지의 거리 -  3em
			\setlist[itemize]{rightmargin=0em}			% 오른쪽 여백  - 4em
			\setlist[itemize]{itemindent=0em} 			% 점 내민 거리 label sep 과 같은면 점위치 까지 내민다
			\setlist[itemize]{listparindent=3em}		% 본문 드려쓰기 간격
	
	
			\setlist[itemize]{ topsep=0.0em, 			%  상단의 여유치
						partopsep=0.0em, 		%  
						parsep=0.0em, 
						itemsep=0.0em, 
						labelwidth=1em, 
						leftmargin=2.5em,
						labelsep=2em,			%  본문에서 라벨 까지의 거리
						rightmargin=0em,		% 오른쪽 여백  - 4em
						itemindent=0em, 		% 점 내민 거리 label sep 과 같은면 점위치 까지 내민다
						listparindent=0em}		% 본문 드려쓰기 간격
	
%			\begin{itemize}
	
		%	-------------------------------------------------------------------------------
		%		Label
		%	-------------------------------------------------------------------------------
			\renewcommand{\labelitemi}{$\bullet$}
			\renewcommand{\labelitemii}{$\bullet$}
%			\renewcommand{\labelitemii}{$\cdot$}
			\renewcommand{\labelitemiii}{$\diamond$}
			\renewcommand{\labelitemiv}{$\ast$}		
	
%			\renewcommand{\labelitemi}{$\blacksquare$}   	% 사각형 - 찬것
%			\renewcommand\labelitemii{$\square$}		% 사각형 - 빈것	
			






% ------------------------------------------------------------------------------
% Begin document (Content goes below)
% ------------------------------------------------------------------------------
	\begin{document}
	
			\dominitoc
			\doparttoc			




			\title{ 길 없는  길 \\ 최인호}
			\author{김대희}
			\date{ 2020년 11월 }
			\maketitle


			\tableofcontents 		% 목차 출력
%			\listoffigures 			% 그림 목차 출력
			\cleardoublepage
			\listoftables 			% 표 목차 출력





		\mdfdefinestyle	{con_specification} {
						outerlinewidth		=1pt			,%
						innerlinewidth		=2pt			,%
						outerlinecolor		=blue!70!black	,%
						innerlinecolor		=white 			,%
						roundcorner			=4pt			,%
						skipabove			=1em 			,%
						skipbelow			=1em 			,%
						leftmargin			=0em			,%
						rightmargin			=0em			,%
						innertopmargin		=2em 			,%
						innerbottommargin 	=2em 			,%
						innerleftmargin		=1em 			,%
						innerrightmargin		=1em 			,%
						backgroundcolor		=gray!4			,%
						frametitlerule		=true 			,%
						frametitlerulecolor	=white			,%
						frametitlebackgroundcolor=black		,%
						frametitleaboveskip=1em 			,%
						frametitlebelowskip=1em 			,%
						frametitlefontcolor=white 			,%
						}



%	================================================================== Part			
	\addtocontents{toc}{\protect\newpage}
	\part{길 없는 길}
	\noptcrule
	\parttoc				


% =============================================================================  등장 인물
%										
% -----------------------------------------------------------------------------										
	\chapter{ 등장 인물}

% ----------------------------------------------------------------------------- 	등장 인물
%										
% -----------------------------------------------------------------------------									
	\section{ 등장인물 }



% ----------------------------------------------------------------------------- 	경허
%										
% -----------------------------------------------------------------------------									
	\section{ 경허 }


\paragraph{9세}


% ----------------------------------------------------------------------------- 	계허
%										
% -----------------------------------------------------------------------------									
	\section{ 계허 }
경허 스님의 스승


% ----------------------------------------------------------------------------- 	역파 스님
%										
% -----------------------------------------------------------------------------									
	\section{ 역파 스님 }

계허 스님의 스승

% ----------------------------------------------------------------------------- 	만화 스님
%										
% -----------------------------------------------------------------------------									
	\section{ 만화 스님 }

게허 스님의 도반

만화는 계허가 잠시 금강산 건봉사에 머물고 있을 때 사귄 친구로 나이는 만화가 휠씬 어렸지만 이미 학문으로서의 명성은 당대 제일이었다.
만화의 속명은 정관준으로 봉화 출신이었는데, 그는 133세에 금강산 건봉사에서 금현 장로를 스승으로 하여 머리를 깍고 중이 되었다.
뒷날, 만링(萬日) 동안 아미타불을 부르며 수도하는 모임인 만일회(萬日會)를 크게 열기도 하였던 당대 제일의 승려였다.
말년에는 대둔사(大屯寺)로 옮겨 그곳에서 오랫동안 주석하다가 1918년 입적한 만화 화상의 법명은 원오(圓悟)였다


% ----------------------------------------------------------------------------- 	아니타
%										
% -----------------------------------------------------------------------------									
	\section{ 아니타 }

\paragraph{}
남전중부사유경의 비유에 나오는 제자중 독수리 잡기를 좋아하는 비구

\paragraph{}
독수리 잡기를 좋아하는 아리카 비구는 나쁜 소견을 가지고 있었다.
그는, 부처님이 언젠가 말씀한 ‘장애’라는 법도 그걸 직접 실행해 보니 그렇게 장애가 되지 않더라고 말했다.
다른 비구들은 그릇된 그의 소견을 고쳐 주려고 토론도 하고 타이르기도 해보았지만 아무 보람이 없었다.
이 말을 전해 들은 부처님을 아리타를 불러 꾸짖으신 후 비구들에게 말씀하셨다.

“어떤 땅꾼이 큰 뱀을 보고 그 몸뚱이나 꼬리를 붙잡았다고 하자.
그때 뱀은 몸을 뒤틀어 붙잡은 손을 물 것이다.
그 때문에 그는 죽거나 죽을 만큼의 고통을 받을 것이다.
그것은 뱀 잡는 방법이 틀렸기 때문이다.
이와 같이 어리석은 사람은 여래의 교법을 배우면서도
가르침의 뜻을 잘 생각하지 않기 때문에 그 진리를 분명하게 알지 못한다.
그런 사람은 토론할 때 말의 권위를 세우려고
곧잘 여래의 교법을 인용하지만 그 뜻을 몰라 난처하게 된다.

그러나 지혜로운 사람은 여래의 가르침을 들으면
그 뜻을 깊이 생각하여 진리를 바르게 알기 때문에 항상 기쁨에 싸여 있다.
이를테면 어떤 땅꾼은 큰 뱀을 보면 곧 막대기로 뱀의 머리를 꼭 누른다.
그때 뱀이 자기를 누르는 손이나 팔을 감는다 할지라도
그 사람은 그 때문에 물려 죽거나
죽을 만큼의 고통을 받지는 않을 것이다.
왜냐하면 그는 뱀 잡는 방법을 잘 알고 있기 때문이다. 

\paragraph{}
비구들이여,
나는 또 너희들에게 집착을 버리도록 하기 위하여 뗏목의 비유를 들겠다.
어떤 나그네가 긴 여행 끝에 바닷가에 이르렀다.
그는 생각하기를 ‘바다 건너 저쪽은 평화로운 땅이다.
그러나 배가 없으니 어떻게 갈까?
갈대나 나무로 뗏목을 엮어 건너가야겠군.’ 하고
뗏목을 만들어 무사히 바다를 건너갔다. 그는 다시 생각하였다.
 ‘이 뗏목이 아니었다면 바다를 건너 올 수 없었을 것이다.
이 뗏목은 내게 큰 은혜가 있으니 메고 가야겠다.
’ 너희들은 어떻게 생각하느냐.
그가 그렇게 함으로써 그 뗏목에 대해 자기 할 일을 다했다고 생각하느냐?”
 
비구들은 하나같이 그렇지 않다고 대답했다.
부처님은 다시 말씀하셨다.
“그러면 그가 어떻게 해야 자기 할 일을 다하게 되겠는가.
그는 바다를 건너고 나서 이렇게 생각해야 할 것이다.
‘이 뗏목으로 인해 나는 바다를 무사히 건너왔다.
다른 사람들도 이 뗏목을 이용할 수 있도록 물에 띄워 놓고 이제 나는 내 갈 길을 가자.’
이와 같이 하는 것이 그 뗏목에 대해서 할 일을 다하게 되는 것이다.

\paragraph{}
 나는 이 뗏목의 비유로써 교법을 배워서 그 뜻을 안 후에는 버려야 할 것이지 결코 거기에 집착할 것이 아니라는 것을 말하였다.
너희들은 이 뗏목처럼 내가 말한 교법까지도 버리지 않으면 안 된다.

 

하물며 법 아닌 것이야 말할 것 있겠느냐.”

% ----------------------------------------------------------------------------- 	년도
%										
% -----------------------------------------------------------------------------									
	\section{ 년도 }

\paragraph{} 
1879년 고종 16년 을묘년


% =============================================================================  배경 지식
%										
% -----------------------------------------------------------------------------										
	\chapter{ 배경 지식}


% ----------------------------------------------------------------------------- 탁발승,행각승
%										
% -----------------------------------------------------------------------------									
	\section{ 행각, 탁발, 걸식 }

불교의 수행 의식 중 하나. 수행자(스님)가 남에게서 음식을 빌어먹는 행위이다. 시주와 비교하면 방향이 반대다. 시주는 신도들이 자발적으로 식량이나 재물을 수행자에게 기부하는 행위 자체나 그러한 행위를 하는 자를 지칭하는 것이며, 탁발은 이 시주를 받기 위해 행하는 수행자의 행동을 말한다.

탁발의 의미는 수행자의 자만과 아집을 버리게 하고, 무소유의 원칙에 따라 끼니를 해결하는 것조차 남의 자비에 의존하는 수행 방식이다. 본래 탁발은 인도 지역의 수행자들이 행하던 전통적인 행위였으며 불교에도 이 영향을 주게 되었다. 석가모니가 불교를 창시한 이후 승려들이 생활을 유지하는 가장 기초적인 방법이었으며, 태국이나 미얀마 등지의 상좌부 불교에서는 승려들이 여전히 이 탁발 행위를 많이 하고 있다.

초기 불교에서는 승려들도 육식을 하였는데 그 이유가 이 탁발 때문이다. 식사 또한 탁발로 100\% 해결하였으므로 얻어먹는 입장에서 사람들이 주는 대로 남기지 않고 먹어야지 거기서 따로 고기를 빼거나 하는 식으로 가려서 먹어서는 안 되었기 때문. 불교에서 채식을 강조하게 된 건 중국의 양무제 시기 이후이다.


한국 불교에서는 다소 부정적으로 보는 개념인데, 이는 한국 불교의 역사와 연관이 있다. 한국 불교는 중국 선종의 영향을 많이 받았는데, 선종에서는 노동 또한 수행의 일종이라고 보고 탁발보다 승려가 스스로 일해서 먹을 것을 마련하는 것을 더 중요한 행위라고 보았다. 이에 대한 선종의 유명한 문구가 '하루 일하지 않으면 하루 먹지 않는다'(一日不作一日不食)이다.

또한 조선 말기부터 사회가 혼란해지면서 불교 조직도 그 체계가 많이 흐트러졌고, 이 과정에서 사이비 승려들이 멋대로 속인들에게 시주를 받아서 재물을 챙기는 행위가 빈번해져서 말이 승려들의 탁발이지 사실상 걸인들의 구걸과 다를 바 없게 되었다.
이 때문에 탁발 행위와 불교 승려에 대한 인식이 매우 나빠지자, 대한불교 조계종 종단에서는 1964년에 아예 탁발 자체를 금지시켰고 신도들의 자발적인 시주만 받도록 하였다.때문에 조계종 승려들은 탁발 행위를 하지 않는다. 대한불교천태종, 태고종 등 다른 제도권 종파에서도 조계종의 선례에 따라 암묵적으로 탁발을 금하고 있기 때문에 길거리 등에서 탁발을 하는 승려는 거의 대부분 승복만 입은 가짜 승려라고 볼 수 있다.





% ----------------------------------------------------------------------------- 무간지옥
%										
% -----------------------------------------------------------------------------									
	\section{ 무간지옥 }

	무간지옥(無間地獄)불교개념용어
     팔열지옥 가운데 고통이 간극이 없이 계속된다는 지옥.
     아비지옥.지옥(阿鼻地獄)분야불교유형개념용어
 \paragraph{정의}
            팔열지옥 가운데 고통이 간극이 없이 계속된다는 지옥.아비지옥.\\
		아비지옥(阿鼻地獄, 산스크리트어: Avīci 아비치) 또는 무간지옥(無間地獄)은 불교의 팔열지옥 중 가장 아래층이며, 
		가장 고통스러운 지옥이다.
		모서리 길이가 2만 요자나(12만-30만 킬로미터)인 정육면체 모양이다.


 \paragraph{내용} 범어(梵語) 아비치(Avici)를 음역하여 아비지옥(阿鼻地獄)이라고도 한다. 팔열지옥(八熱地獄)의 하나로서, \textbf{무간이라고 한 것은 그곳에서 받는 고통이 간극(間隙)이 없이 계속되기 때문이다}.우리나라 불교에서는 이 지옥이 가장 대표적인 지옥으로 알려져 있으며, 불교 경전 및 우리나라 고승들의 저술에서도 그 이름이 자주 나타나고 있다. 이 지옥은 남섬부주 아래 4만 유순(由旬)이 되는 지하에 있다.
\paragraph{}
여러 경전에 묘사된 이 지옥의 고통 받는 모습으로는 옥졸이 죄인을 잡아 가죽을 벗기고, 그 벗겨낸 가죽으로 죄인의 몸을 묶어 불수레에 실은 뒤 타오르는 불길 속에 넣어 몸을 태우며, 야차들은 큰 쇠창을 불에 달구어 죄인의 몸을 꿰거나 입·코·배 등을 꿰어 공중에 던지기도 한다. 또, 철로 만들어진 매가 죄인의 눈을 파먹는 등 극심한 형벌을 받게 된다.
 \paragraph{}
무간지옥에 떨어져서 이러한 고통을 받게 되는 까닭은 부모나 덕이 높은 스승을 죽이는 등의 오역죄(五逆罪) 중 어느 하나를 범하거나, 인과(因果)를 무시하고 절이나 탑을 무너뜨리며, 성중(聖衆)을 비방하거나 수행하지 않고 시주가 주는 음식만을 먹는 경우라고 한다.
 \paragraph{}
이 지옥의 고통 받는 모습은 사찰 명부전(冥府殿) 안의 시왕탱화(十王幀畫) 속에 묘사되어 있는 경우가 많으며, 여러 문학 작품이나 민간 설화에도 이에 대한 표현이 나타나고 있다.
 \paragraph{          참고문헌}
        『종교사화(宗敎史話)』(이기영,한국불교연구원,1978)
          집필자집필
            (1997년)정병조경기도 성남시 분당구 하오개로 323 한국학중앙연구원 [13455]대표전화 031-709-8111Copyright the Academy of Korean Studies. All Right Reserved.Family site한국향토문화전자대전한국학자료센터한국역대인물종합시스템장서각 디지털아카이브
				

출처: 한국민족문화대백과사전(무간지옥(無間地獄))


% ----------------------------------------------------------------------------- 불목하니
%										
% -----------------------------------------------------------------------------									
	\section{ 불목하니}

불목하니는 사찰에서 땔나무를 베고 물을 긷는 사내 종노를 뜻하였던 한국 고유의 직업이다. 
숙종 시기를 전후한 조선 시대 중기에서 대한제국 멸망 이후의 일제 강점기와 6·25 한국 동란 직전 시기까지 한반도의 사찰에서는 흔히 있었던 직업이다.




% ----------------------------------------------------------------------------- 논어
%										
% -----------------------------------------------------------------------------									
	\section{ 아승기겁}
		아승기겁(阿僧祇劫)불교개념용어
             헤아릴수 없는 오랜 시간을 가리키는 불교용어.
                무량겁.확대하기축소하기프린트URL의견제시트위터페이스북의견제시항목명아승기겁이메일올바른 형식의 이메일을 입력해 주세요.의견10자 이상 상세히 작성해 주세요.첨부파일파일선택이칭무량겁(無量劫)분야불교유형개념용어
 \paragraph{정의}
            헤아릴수 없는 오랜 시간을 가리키는 불교용어.무량겁.
 \paragraph{내용}
아승기겁(阿僧祇劫)은 헤아릴 수 없는 시간을 뜻하는 산스크리트어 asamkhyeya-kalpa의 음역어이다. 팔리어로는 asankheyya-kalpa라고 하며, 무량겁(無量劫)으로 의역된다. 아승기는 헤아릴수 없는 무한히 큰 수를 뜻하는데 아승(阿僧)·아승가(阿僧伽)로 음역되거나, 불가산계(不可算計)·무량수(無量數)·무앙수(無央數)로 의역된다.아승기는 인도의 숫자 단위중 하나인 60가지의 수목(數目) 단위 중 52번째 단위이며, 경전마다 그 수의 크기가 다르게 나타난다. 
 \paragraph{}
『화엄경』「승기품 」은 120종류의 대수(大數)중에 백천(百千)의 자승[제곱]을 1구지(俱胝)라 할 때 구지를 반복적으로 100번 이상 곱한 것이 아승기라고 한다. 『대지도론』은 아승기를 무수(無數)로 의역했는데, 아승기란 더 이상 헤아릴 수 없는 단위의 숫자를 의미한다고 하였다.아승기겁은 무한한 숫자를 뜻하는 '아승기'와 시간을 뜻하는 '겁'이 결합하여 계산할 수 없는 정도로 무한한 긴 시간을 의미한다. 이 용어는 무량한 시간동안 윤회를 반복하거나, 해탈을 위해 무량한 시간동안 수행하는 것을 강조할 때 쓰이는 표현이다. 『장아함경』 제1권 「대본경」에 "나는 헤아릴 수 없이 많은 아승기겁 이전부터 부지런히 애쓰며 나태하지 않고 최상의 수행을 해오다가 이제야 비로소 이렇게 성취하기 어려운 법을 얻었다"는 표현이 나타난다. 『대비바사론』 제177권은 삼아승기겁으로 겁아승기야·생아승기야·묘행아승기야를 제시하고, 이런 긴 시간동안 수행을 해야 최상의 깨달음을 증득한다고 하였다.
 \paragraph{ 참고문헌}

        장아함경(長阿含經)대지도론(大智度論)화엄경(華嚴經)대비바사론(大毘婆沙論)일체경음의(一切經音義)


출처: 한국민족문화대백과사전(아승기겁(阿僧祇劫))


% ----------------------------------------------------------------------------- 논어
%										
% -----------------------------------------------------------------------------									
	\section{ 논어 학이 16장}


(원문)

子曰, 不患人之不己知, 患不知人也.

자왈, 불환인지부기지, 환불지인야.



(해석)

공자께서 말씀하셨다. "남이 나를 알아주지 않음을 걱정하지 말고, 내가 남을 알지 못함을 걱정해야 한다."



(풀이)

 짧지만 날카롭지 않은가? 우리는 왜 내 말을 못 알아듣냐며 펄쩍펄쩍 뛰지만 그 전에 남이 하는 말에 정말 귀 기울여 본 적이 얼마나 있던가? 



 학이편의 마지막을 장식하는 이 짤막한 장은, 첫 장의 세 번째 구절인 '남이 나를 알아주지 않더라도 섭섭해하지 않으니 어찌 군자가 아닌가!' 와 같은 뜻의 말이다. 여기 이 장이 놓인 것은 결코 우연한 일이 아니리라. 전해져 내려오던 논어의 편들을 엮은, 그 이름을 알 수 없는 편집자도 '남이 알아주지 않는' 위치에 있으며 그러함에도 '섭섭해하지 않고' 자기 영혼의 스승인 공자를 바라보고 있는 것인지도 모른다. 


% ----------------------------------------------------------------------------- 장자 제26편 외물
%										
% -----------------------------------------------------------------------------									
	\section{ 장자 제26편 외물}


% =============================================================================  시
%										
% -----------------------------------------------------------------------------										
	\chapter{ 시}


% ----------------------------------------------------------------------------- 계명숙
%										
% -----------------------------------------------------------------------------									
	\section{ 계명숙 }

		고려때 원나라의 사신으로온 계명숙의 시

계명숙(季明叔) ;원나라에서고려사신으로와서 천안의객관에서지은 시

\begin{quote}

말 탄 길손이 저물녘에 천안에 와서 \\
문안으로 들어가 말에서 내려 \\
한가로이 서성거리네  \\
빈뜰 고요하여 만뢰가 쥐죽은 듯한데  \\
낙엽만이 쓸쓸히 난간을 울리네  \\
푸른 하늘엔 구름없어 맑기가 씻은 듯하고  \\
밤빛에 맺힌 이슬 반짝이는데  \\\
호상에 홀로 앉아 잠 못 이루니  \\
달은 날아오고 바람이 차갑구나  \\

\end{quote}


\begin{verbatim}

말 탄 길손이 저물녘에 천안에 와서 
문안으로 들어가 말에서 내려
한가로이 서성거리네
빈뜰 고요하여 만뢰가 쥐죽은 듯한데
낙엽만이 쓸쓸히 난간을 울리네
푸른 하늘엔 구름없어 맑기가 씻은 듯하고
밤빛에 맺힌 이슬 반짝이는데
호상에 홀로 앉아 잠 못 이루니
달은 날아오고 바람이 차갑구나

\end{verbatim}


% =============================================================================  책
%										
% -----------------------------------------------------------------------------										
	\chapter{ 책}


% ----------------------------------------------------------------------------- 	구덕도서관
%										
% -----------------------------------------------------------------------------									
	\section{ 구덕도서관}




			\begin{itemize}[	topsep=0.0em,
							itemsep=0.0em,
							leftmargin=5em, 
							labelsep=1em ]
			\item					최인호 길없는 길 여백
			\item					4판 2쇄 발행 : 2013년 12월 26일 \\ 

			\item			813.62 14 1 거문고의 비밀
			\item			813.62 14 2 
			\item			813.62 14 3
			\item			813.62 14 4 하는가의 방랑

			\end{itemize}	






%	================================================================== Part			
	\addtocontents{toc}{\protect\newpage}
	\part{길 없는 길}
	\noptcrule
	\parttoc				



% =============================================================================  거문고의 비밀
%										
% -----------------------------------------------------------------------------										
	\chapter{ 거문고의 비밀}

% ----------------------------------------------------------------------------- 	거문고의 비빌
%										
% -----------------------------------------------------------------------------									
	\section{ 거문고의 비밀 }

% ----------------------------------------------------------------------------- 	대발심
%										
% -----------------------------------------------------------------------------									
	\section{ 대발심 }


% ----------------------------------------------------------------------------- 	내 마음의 왕국 
%										
% -----------------------------------------------------------------------------									
	\section{ 내 마음의 왕국 }


% =============================================================================  불타는 집
%										
% -----------------------------------------------------------------------------										
	\chapter{ 불타는 집}

% ----------------------------------------------------------------------------- 	내 마음의 왕국 
%										
% -----------------------------------------------------------------------------
	\section{ 선의 강물 }

% ----------------------------------------------------------------------------- 	불 타는 집
%										
% -----------------------------------------------------------------------------
	\section{ 불타는 집 }

% ----------------------------------------------------------------------------- 	꺼지지 않는 등불
%										
% -----------------------------------------------------------------------------
	\section{ 꺼지지 않는 등불 }



% =============================================================================  생각의 화살
%										
% -----------------------------------------------------------------------------										
	\chapter{ 생각의 화살}


% ----------------------------------------------------------------------------- 	생각의 화살
%										
% -----------------------------------------------------------------------------
	\section{ 생각의 화살}

% ----------------------------------------------------------------------------- 	세개의 달
%										
% -----------------------------------------------------------------------------
	\section{ 세개의 달}

% =============================================================================  하늘가의 방랑객
%										
% -----------------------------------------------------------------------------										
	\chapter{ 하늘가의 방랑객}


% ----------------------------------------------------------------------------- 	진흙소의 울음
%										
% -----------------------------------------------------------------------------
	\section{ 진흙소의 울음}

% ----------------------------------------------------------------------------- 	뒤에 오는 사람
%										
% -----------------------------------------------------------------------------
	\section{ 뒤에 오는 사람 }

% ----------------------------------------------------------------------------- 	하늘가의 방랑객
%										
% -----------------------------------------------------------------------------
	\section{ 하늘가의 방랑객}


% ----------------------------------------------------------------------------- 	길없는 길
%										
% -----------------------------------------------------------------------------
	\section{ 길없는 길}






% ------------------------------------------------------------------------------
% End document
% ------------------------------------------------------------------------------
\end{document}


	\href{https://www.youtube.com/watch?v=SpqKCQZQBcc}{태양경배자세A}
	\href{https://www.youtube.com/watch?v=CL3czAIUDFY}{태양경배자세A}


https://docs.google.com/spreadsheets/d/1-wRuFU1OReWrtxkhaw9uh5mxouNYRP8YFgykMh2G_8c/edit#gid=0
+

https://seoyeongcokr-my.sharepoint.com/:f:/g/personal/02017_seoyoungeng_com/Ev8nnOI89D1LnYu90SGaVj0BTuckQ46vQe1HiVv-R4qeqQ?e=S3iAHi