%	------------------------------------------------------------------------------
%
%			작성 : 2020년 7월 14일 첫 작업
%
%

%	\documentclass[25pt, a1paper]{tikzposter}
%	\documentclass[25pt, a0paper, landscape]{tikzposter}
%	\documentclass[25pt, a0paper ]{tikzposter}
	\documentclass[	25pt, 
							a0paper, 
							portrait, %
							margin=0mm, %
							innermargin=10mm,  		%
							blockverticalspace=4mm, %
							colspace=5mm, 
							subcolspace=0mm
							]{tikzposter}
%	\documentclass[25pt, a1paper]{tikzposter}
%	\documentclass[25pt, a1paper]{tikzposter}
%	\documentclass[25pt, a1paper]{tikzposter}

% 	12pt  14pt 17pt  20pt  25pt
%
%	a0 a1 a2
%
%	landscape  portrait
%

	%% Tikzposter is highly customizable: please see
	%% https://bitbucket.org/surmann/tikzposter/downloads/styleguide.pdf

	%	========================================================== 	Package
		\usepackage{kotex}						% 한글 사용


%% Available themes: see also
%% https://bitbucket.org/surmann/tikzposter/downloads/themes.pdf
%	\usetheme{Default}
%	\usetheme{Rays}
%	\usetheme{Basic}
	\usetheme{Simple}
%	\usetheme{Envelope}
%	\usetheme{Wave}
%	\usetheme{Board}
%	\usetheme{Autumn}
%	\usetheme{Desert}

%% Further changes to the title etc is possible
%	\usetitlestyle{Default}			%
%	\usetitlestyle{Basic}				%
%	\usetitlestyle{Empty}				%
%	\usetitlestyle{Filled}				%
%	\usetitlestyle{Envelope}			%
%	\usetitlestyle{Wave}				%
%	\usetitlestyle{verticalShading}	%


%	\usebackgroundstyle{Default}
%	\usebackgroundstyle{Rays}
%	\usebackgroundstyle{VerticalGradation}
%	\usebackgroundstyle{BottomVerticalGradation}
%	\usebackgroundstyle{Empty}

%	\useblockstyle{Default}
%	\useblockstyle{Basic}
%	\useblockstyle{Minimal}		% 이것은 간단함
%	\useblockstyle{Envelope}		% 
%	\useblockstyle{Corner}		% 사각형
%	\useblockstyle{Slide}			%	띠모양  
	\useblockstyle{TornOut}		% 손그림모양


	\usenotestyle{Default}
%	\usenotestyle{Corner}
%	\usenotestyle{VerticalShading}
%	\usenotestyle{Sticky}

%	\usepackage{fontspec}
%	\setmainfont{FreeSerif}
%	\setsansfont{FreeSans}

%	------------------------------------------------------------------------------ 제목

	\title{ 금정불교대학 19년 야간 능인회 }

	\author{ 2020년 7월 15일 기준 }

%	\institute{서영엔지니어링}
%	\titlegraphic{\includegraphics[width=7cm]{IMG_1934}}

	%% Optional title graphic
	%\titlegraphic{\includegraphics[width=7cm]{IMG_1934}}
	%% Uncomment to switch off tikzposter footer
	% \tikzposterlatexaffectionproofoff

\begin{document}

	\maketitle[
					width=841mm,
					linewidth = 2mm,
					innersep=4mm,
%					titletotopverticalspace=0mm, %
%					titletoblockverticalspace=0mm, %
					titletextscale =4, 
				]

		%		a0  841 - 1189
		%		a1  594 - 841
		%		a2  420 - 594


	\begin{columns}

		\column{0.5}

%	------------------------------------------------------------------------------ 조직  }

			\block[
						titleoffsetx =0mm,	 	%
						titleoffsety=0mm,	 	% 
						bodyoffsetx=00mm,	%
						bodyoffsety=0mm	 	%
						titlewidthscale=2, 
						bodywidthscale=1,
						titleleft, 
%						titlecenter, 
%						titleright,
						bodyverticalshift=0mm  % 제목과 본과의 간격
						roundedcorners=50, 
						linewidth=1mm,
						titleinnersep=10mm, 
%						bodyinnersep=0mm
					]
{■  조직  }
			{
					\setlength{\leftmargini}{6em}
					\setlength{\labelsep} {1em}
				\begin{LARGE}
					\begin{itemize}
					\item  [회장] 양도근
					\item  [사무]김대희
					\item  [총무] 하승희
					\item  [총무] 최윤교
					\item  [위원] 여종한
					\end{itemize}
				\end{LARGE}
			}



%	------------------------------------------------------------------------------ 조직  }
			\block{■  조직 : 감사 }
			{

%			\innerblock[options]{Heading}{Text}, 

					\setlength{\leftmargini}{6em}
					\setlength{\labelsep} {1em}
				\begin{LARGE}
					\begin{itemize}
					\item  [감사] 정대성
					\item  [감사] 조현진
					\end{itemize}
				\end{LARGE}
			}



%	------------------------------------------------------------------------------ 조직  }
			\block{■  조직 : 부회장 }
			{
					\setlength{\leftmargini}{9em}
					\setlength{\labelsep} {1em}
				\begin{LARGE}
					\begin{itemize}
					\item  [수석부회장] 정일현
					\item  [부회장] 
					\item  [부회장] 
					\item  [부회장] 
					\item  [부회장] 
					\item  [부회장] 

					\end{itemize}
				\end{LARGE}
			}


%	------------------------------------------------------------------------------ 통장 }
			\block{■  통장 : 일반 회비  }
			{
					\setlength{\leftmargini}{8em}
					\setlength{\labelsep} {1em}
				\begin{LARGE}
					\begin{itemize}
					\item [제목] 일반회비
					\item [은행] 농협
					\item [계좌번호] 312 - 0200 - 5624 - 81
					\item [예금주] 양도근
					\item [관리]
					\end{itemize}
				\end{LARGE}
			}


%	------------------------------------------------------------------------------ 통장 }
			\block{■  통장 : 임원 회비  }
			{
					\setlength{\leftmargini}{8em}
					\setlength{\labelsep} {1em}
				\begin{LARGE}
					\begin{itemize}
					\item [제목] 임원 회비
					\item [은행] 농협
					\item [계좌번호] 302 - 1372 - 7281 - 11
					\item [예금주] 양도근
					\item [관리]
					\end{itemize}
				\end{LARGE}
			}


%	------------------------------------------------------------------------------ 장인철 }
			\block{■  장 인철 }
			{
					\setlength{\leftmargini}{4em}
					\setlength{\labelsep} {1em}
				\begin{LARGE}
					\begin{itemize}
					\item [이름] 장인철
					\item [전번] 010 8936 8767
					\item [주소] 천금프린트 동래구 복천동 374-4
					\end{itemize}
				\end{LARGE}
			}



%	------------------------------------------------------------------------------ 오종현 }
			\block{■  오 종현 }
			{
					\setlength{\leftmargini}{4em}
					\setlength{\labelsep} {1em}
				\begin{LARGE}
					\begin{itemize}
					\item [이름] 오 종현
					\item [전번] 010 2999 6131
					\item [주소] 좋은사람들 재가복지센터 동래구 중앙대로 1301번길 22
					\end{itemize}
				\end{LARGE}
			}


	%	====== ====== ====== ====== ====== 
		\column{0.5}

%	------------------------------------------------------------------------------ 무료급식
			\block{■  행사 : 무료급식 8월 20일 (목) }
			{
					\setlength{\leftmargini}{4em}
					\setlength{\labelsep} {1em}
				\begin{LARGE}
					\begin{itemize}
					\item [제목] 금정불교대학 능인회 무료급식
					\item [일자] 8월 20일 (목)
					\item [참석] 
					\end{itemize}
				\end{LARGE}
			}



	\end{columns}




\end{document}


		\begin{huge}
		\end{huge}

		\begin{LARGE}
		\end{LARGE}

		\begin{Large}
		\end{Large}

		\begin{large}
		\end{large}

